% \documentclass[main.tex]{subfiles}
\documentclass[12pt,a4paper,twoside]{article}
\usepackage[utf8]{inputenc}

\title{Introduction}
\author{Team 5 - STO \thanks {Carlos Nava, Abraham Escajadillo}}
\date{Spring 2020}

\begin{document}
\maketitle

The belief that mathematical location modeling can identify "optimal" location patterns rests on the basis that some realistic objective can be identified and by some measure quantified.\\\\
The maximal covering location problem (MCLP) is a challenging problem with numerous applications in practice.\\\\
It has been shown that the p‐median problem, the location set‐covering and the maximal covering location problems are important facility location models. This paper gives  a  historical  perspective  of  the development of these models and identifies the theoretical links between them. It is shown that the 
maximal covering location problem can be structured and solved as a p‐median problem in addition to the several approaches already developed. Computational experience for several maximal covering location problems is given.\\\\
The use of a maximal service distance as a measure of the value of a given locational configuration has been discussed at length by Toregas and ReVelle(European Journal of Operational Research 88 (1996) 114-123 European Journal of Operational Research 88 (1996) 114-123 ) who show that it is an important surrogate measurement for the value of a given locational configuration. For a given location solution, the maximum distance which any user would have to travel to reach a facility would reflect the worst possible performance of the system. In the regional location of emergency facilities such as fire stations or ambulance dispatching stations, the concept of the maximal service distance is well established.\\\\
Some applications to the MCLP problem is in wifi antennas; this problem allow to extend the rate of signal of the internet in a certain place.\\\\
The MCLP was proposed by ReVelle and Church in 1974 \\(Ing. Cynthia Porras Nodarse, 2018, \\http://www.informaticahabana.cu/sites/default/files/ponencias2018/CCI08.pdf).\\ A study by Murray in 2016 revealed that since the first publication of the MCLP until 2015 it has had more than 1550 citations in academic literature and is considered as one of the pillars between location problems. He currently has more than 2,000 citations, which represents an increase in research on the problem. Several of the applications of this problem are: the location of the patrol stations in the police, organization and distribution of ambulances, location of air medical brigades, location of cellular stations, location of mobile sensors, location of fire stations, location of places for the preservation of the habitat of species and camera locations in an urban traffic network, among others.\\\\
The MCLP has a great relationship with the SCLP. In the SCLP pursues the goal of finding the number minimum of facilities that cover all demand. Therefore, when the number of facilities to be located in the MCLP is greater than the number minimum of installations, there is no point in applying the MCLP, since the answer to the problem is already given by the SCLP. The objective of this work is to apply the MCLP, using metaheuristic algorithms, to the location wi-fi antennas for internet connection in order to offer this service to the largest possible population.

\end{document}