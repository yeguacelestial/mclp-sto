
\documentclass[11pt, a4paper]{article}

\usepackage[T1]{fontenc}
\usepackage{mathpazo}
\usepackage{graphicx}\graphicspath{{gfx/}}
\usepackage[margin=1in]{geometry}
\usepackage{setspace}
\usepackage{caption,subcaption}
\usepackage[sort&compress, numbers]{natbib}
\usepackage{hyperref}

\captionsetup{font={small, stretch=1}, labelfont=bf}
\renewcommand{\figurename}{Fig.}
\setlength{\bibsep}{3pt}
\newcommand{\Title}[1]{{\LARGE \centering \hrulefill\\ \textbf{#1}\\ \hrulefill}}
%==================================================================

\begin{document}

\pagenumbering{roman}
\pagestyle{plain}


\onehalfspacing
\setcounter{page}{1}
\pagenumbering{arabic}
\Title{Computational Experience with the Maximal Location Covering Problem}

\section*{Abstract}
{\small \singlespacing
	On this paper, you can find heuristic approaches applied to the Maximal Covering Location Problems; this project could be useful to decision makers that want to generate a solution and try to improve a given solution as well in a short amount of time.  
}

\section{Introduction}\label{sec:intro}
Facility Location is a branch of Operations Research. This category of combinatorial optimization problems often deals with problems that seek to select the placement of a facility (often, from a given list of possibilities) that meets the best of certain constraints. These problems often consists of minimizing the total weighted distances from supplies and customers, and weights are representative of the difficulty of transporting materials. 

Then, we have the Maximal Covering Location Problem. This problem is a derivative of Facility Location sort; however, this problems seeks to maximize the customers covered by a given set of facilities,instead of minimizing the distance between customers and supplies. 

The Maximal Covering Location Problem (MCLP), is a classic problem in location analysis with    applications in a good number of fields, such as health care, emergency planning, ecology, statistical classification, homeland security, etc.

The objective of the MCLP is to maximize the population covered by a given set of facilities; however, this could work as an analogy to another problems that involve "supply" and "demand" points to maximize the demand as much as possible.

Formally, this problem is described in Church and ReVelle 1974 as follows:
"The maximal covering location problem seeks the maximum population which can be served within a stated service distance or time given a limited number of facilities."

In this paper, the MCLP is analysed and solved on random instances generated within a certain range, by an implemented instance generator. A Constructive Heuristic approach was used to generate a solution, and a Local Search algorithm was implemented to try to improve the feasible solution generated by the constructive. Overall, the effectiveness and computational time of the implemented algorithms on small, medium and large instances were analysed. More details are given in the following sections.

\section{Problem description and example}\label{sec:lit}
According to the model defined on Church and ReVelle 1974, this problem is defined on a network of nodes and arcs. 

A mathematical formulation of this problem can be stated as maximizing Eq. ~\ref{eq: math model}:

\begin{equation}\label{eq: math model}
	z = \sum_{i \in I}y_i
\end{equation}
subject to:

\begin{equation}\label{eq: r1}
	x_{j} \in (0,1), j \in J
\end{equation}

\begin{equation}\label{eq: r2}
	y_{i} \in (0,1), i \in I
\end{equation}

\begin{equation}\label{eq: r3}
	\sum_{j \in J}x_j = K
\end{equation}

\begin{equation}\label{eq: r4}
	\sum_{j \in N_i} x_j \geq y_i, N_i = \big\{ j \in J: d_{ij} \leq r \big\}
\end{equation}

\subsection{Equation}
One of the greatest motivating forces for Donald Knuth when he began developing the original TeX system was to create something that allowed simple construction of mathematical formulae, while looking professional when printed. The fact that he succeeded was most probably why TeX (and later on, LaTeX) became so popular within the scientific community. Typesetting mathematics is one of LaTeX's greatest strengths. It is also a large topic due to the existence of so much mathematical notation. As Eq.~\ref{eq: momentum diff} shows how to label any equation in latex. 

The velocity, v ($v=d/t$) is ....
%
\begin{equation}\label{eq: momentum diff}
\nu = \frac{\mu}{\rho}
\end{equation}

\subsection{How to cite?}
Citations are referred in the text using \verb|\citet| command which produces citations
as though they are part of the text. In order to say somebody did this work as a part of a line use:~ Voet and Voet have done extensive work on .... This will produce~\citet{voet2011biochemistry} have done extensive work on .... Alternately citations can appear in parenthesis. The command~\citep{voet2011biochemistry} is used to automatically put the citations in parenthesis. As an example consider the extensive work done in the area of book writing~\citep{seifert1991shape}.
Article~\citep{sircar1972adsorption,keh1995particle} and collection of work~\citep{seifert1995morphology} are the other examples.

\section{Heuristics}\label{sec:exp}

\subsection{Description of Constructive Heuristic}
An experiment is a procedure carried out to support, refute, or validate a hypothesis. Experiments provide insight into cause-and-effect by demonstrating what outcome occurs when a particular factor is manipulated. Experiments vary greatly in goal and scale, but always rely on repeatable procedure and logical analysis of the results. There also exists natural experimental studies.

A child may carry out basic experiments to understand gravity, while teams of scientists may take years of systematic investigation to advance their understanding of a phenomenon. Experiments and other types of hands-on activities are very important to student learning in the science classroom. Experiments can raise test scores and help a student become more engaged and interested in the material they are learning, especially when used over time. Experiments can vary from personal and informal natural comparisons (e.g. tasting a range of chocolates to find a favorite), to highly controlled (e.g. tests requiring complex apparatus overseen by many scientists that hope to discover information about subatomic particles). Uses of experiments vary considerably between the natural and human sciences. Fig.~\ref{fig:parabolic} is the parabolic plot.

\begin{figure}
	\centering
	% \includegraphics[width=0.45\linewidth]{parabolic}
	\caption{Caption of fig. 1.}
	\label{fig:parabolic}
\end{figure}

\subsubsection{First subsubsection}
An experiment usually tests a hypothesis, which is an expectation about how a particular process or phenomenon works. However, an experiment may also aim to answer a ``what-if'' question, without a specific expectation about what the experiment reveals, or to confirm prior results. If an experiment is carefully conducted, the results usually either support or disprove the hypothesis. According to some philosophies of science, an experiment can never "prove" a hypothesis, it can only add support. On the other hand, an experiment that provides a counterexample can disprove a theory or hypothesis, but a theory can always be salvaged by appropriate ad hoc modifications at the expense of simplicity. An experiment must also control the possible confounding factors—any factors that would mar the accuracy or repeatability of the experiment or the ability to interpret the results. Confounding is commonly eliminated through scientific controls and/or, in randomized experiments, through random assignment.

In engineering and the physical sciences, experiments are a primary component of the scientific method. They are used to test theories and hypotheses about how physical processes work under particular conditions (e.g., whether a particular engineering process can produce a desired chemical compound). Typically, experiments in these fields focus on replication of identical procedures in hopes of producing identical results in each replication. Random assignment is uncommon.

\paragraph{Paragraph tile.}
According to his explanation, a strictly controlled test execution with a sensibility for the subjectivity and susceptibility of outcomes due to the nature of man is necessary.

\subsection{Description of Local Search}
In engineering and the physical sciences, experiments are a primary component of the scientific method. They are used to test theories and hypotheses about how physical processes work under particular conditions (e.g., whether a particular engineering process can produce a desired chemical compound). Typically, experiments in these fields focus on replication of identical procedures in hopes of producing identical results in each replication. Random assignment is uncommon.

\section{Experiments}\label{sec:conc} 
Conclusions show readers the value of your completely developed argument or thoroughly answered question. Consider the conclusion from the reader's perspective. At the end of a paper, a reader wants to know how to benefit from the work you accomplished in your paper. 

\section{Conclusions}
Conclusions show readers the value of your completely developed argument or thoroughly answered question. Consider the conclusion from the reader's perspective. At the end of a paper, a reader wants to know how to benefit from the work you accomplished in your paper. 

\appendix
\section{Supporting information}
The purpose of the supporting information is to enable authors to provide and archive supporting information such as data tables, method information, figures, video, or computer software, in digital formats so that other scientists can use it.

\small \singlespacing
\bibliographystyle{unsrtnat} 
\bibliography{mybib}

\end{document}









 





















